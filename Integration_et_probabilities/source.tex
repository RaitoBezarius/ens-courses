% !TeX spellcheck = fr-FR


\documentclass[10pt,a4paper,notitlepage ]{article}
\usepackage[utf8]{inputenc}
\usepackage[T1]{fontenc}
\usepackage[french]{babel}
\usepackage{amsmath}
\usepackage{amsfonts}
\usepackage{amssymb}
\usepackage{graphicx}
\usepackage{dsfont}
\usepackage{tcolorbox}
\usepackage{stmaryrd}


\title{Intégration et probabilités}
\date{}

\newenvironment{definition}{
	
	\textbf{Définition : }
}
{}

\newcounter{th}
\newenvironment{theorem}[1][]{
\refstepcounter{th}
\begin{tcolorbox}
	\textbf{Théorème \theth \ #1}
	
	
}{\end{tcolorbox}}

\newenvironment{propriete}{
	\begin{tcolorbox}
		\textbf{Propriété : }
}
{\end{tcolorbox}}

\newenvironment{demo}{

	\textbf{Démonstration :}
}{\begin{flushright}
	$\square$
\end{flushright}
}

\newenvironment{exo}{
	
	\textbf{Exercice :} }{}

\newenvironment{exemple}{
	
	\textbf{Exemple :} }{}

\newenvironment{corollaire}{
	\begin{tcolorbox}
		\textbf{Corollaire : }
	}
	{\end{tcolorbox}}

\newenvironment{lemme}{
	\begin{tcolorbox}
		\textbf{Lemme : }
	}
	{\end{tcolorbox}}

\newenvironment{rem}{
	
	\textbf{Remarque :}}{}

\begin{document}
	\maketitle
	\part*{Introduction}
	Références : 
	\begin{itemize}
		\item \textsc{Billingsley}, \textit{Probabiliy and measure}
		\item \textsc{Kolmogorov \& Fomin}, tome 2 
	\end{itemize}
	Motivations :
	\begin{itemize}
		\item Définir la longueur d'une partie de $\mathbb R$
		\item Définir l'aire d'une partie de $\mathbb R ^2$
		\item  Définir $\int f \mathrm dx$ pour $f : \mathbb R ^d \rightarrow \mathbb R$
		\item Définir, préciser la notion mathématique décrivant une suite infinie de jets de dés
	\end{itemize}
Par exemple :
\begin{itemize}
	\item Si $f : \mathbb R \rightarrow \mathbb R$, on peut définir $\int f$ comme l'aire algébrique définie par le graphe de $f$. Ainsi, définir une aire permet de définir une intégrale
	\item De même, $\lambda(A) = \mathds 1_A$ avec $\mathds 1_A(x)=1$ ssi $x \in A$. Donc définir une intégrale revient à définir une mesure.
	\item Tirer un nombre au hasard dans $[\,0,1]\,$, cela revient à tirer au hasard la suite de ses décimales au D10, car on mesure une partie de $\{ 0, 1, \dots 9 \}^\mathbb N$
 \end{itemize}

On se demande alors comment définir la surface d'une partie du plan.

Méthode 1 : à la Riemann. On approxime avec un quadrillage. On compte le nombre de carrés qui intersectent l'ensemble considéré, puis on conclut en passant à la limite quand le côté du quadrillage tend vers $0$.

Méthode 2 : on pose $\lambda(A) := \underset{(R_i)}{\inf} \sum_{i=1}^{\infty}  \lambda(R_i)$ où $R_i$ est une suite de rectangles recouvrant $A$.

À noter : les deux méthodes ont des cas pathologiques différents.

\part*{Ensembles dénombrables}

\begin{definition}
	Un ensemble est dénombrable ssi il est en bijection avec $\mathbb N$
\end{definition}
\begin{propriete}
	Toute partie d'un ensemble dénombrable est au plus dénombrable
\end{propriete}
\begin{demo}
	On pose $x : \mathbb N \rightarrow X, Y \subset X$. Si $Y$ n'est pas fini :
	\begin{align*}
	&i_1 = \min \{i\in\mathbb N, x_i \in Y \} \\
	&\dots  \\
 	&i_n = \min \{ i\in \mathbb N, x_i\in Y \setminus\{x_1, \dots,x_{n-1}\}\}
 	\end{align*} 
	Ainsi, $k \mapsto x_{n_k}$ est une bijection de $\mathbb N$ vers $Y$.
\end{demo}

\begin{propriete}
	L'image d'une suite est au plus dénombrable.
\end{propriete}
\begin{demo}
	On note $x:\mathbb N \rightarrow X$ une suite.
On crée de manière analogue une sous-suite injective de $x$ de même image que $x$ (sauf si $f(x(\mathbb N))$ est fini).
\end{demo}
\begin{propriete}
	$\mathbb N \times \mathbb N$ est dénombrable.
\end{propriete}
\begin{demo}
	$(n_1,n_2) \mapsto 2^{n_1} (2n_2+1) -1$ est une bijection $\mathbb N^2 \rightarrow \mathbb N$.
\end{demo}

\begin{propriete}
	Une réunion au plus dénombrables d'ensembles au plus dénombrable est au plus dénombrable.
\end{propriete}

\begin{demo}
	On traite le cas "union dénombrable d'ensembles dénombrables".
	
	Soit $A_i$ des parties dénombrables d'un ensemble $X$.
	Pour tout $i$, il existe $b_i : \mathbb N \rightarrow A_i$ bijection. (nb : ceci requiert en fait l'axiome du choix dénombrable)
	Alors$ \begin{aligned}
		(i,j) &\mapsto b_i(j) \\
		\mathbb N ^2 & \rightarrow \underset{i}{\bigcup} A_i
	\end{aligned}$ est surjective.

Donc $\underset{i}{\bigcup}A_i$ est au plus dénombrable.

Or $\underset{i}{\bigcup}A_i \supset A_i$.

Donc $\underset{i}{\bigcup} A_i$ est dénombrable.
\end{demo}

\begin{propriete}
	Si $X$ est dénombrable, $\mathcal P(X)$ ne l'est pas.
	
	Plus généralement, quel que soit $X$, $X$ et $\mathcal P(X)$ ne sont jamais en bijection (théorème de Cantor).
\end{propriete}

\begin{demo}
	Supposons qu'il existe $x: \begin{aligned} X & \rightarrow \mathcal P(X) \\
		x & \mapsto A_x \end{aligned}$ une bijection.
	
	Considérons $B := \{x, x\notin A_x\}$. Comme $x$ est une bijection, il existe $y \in X$ tel que $B = A_y$.
	
	Question : a-t-on $y\in B$. On arrive à un paradoxe type Russel.
\end{demo}

\begin{exo}
\begin{itemize}	
	\item $\{0,1\}^\mathbb N$ est non dénombrable.
	
	\item $\mathbb R$ est non dénombrable.
	
\end{itemize}
\end{exo}

\part*{$\limsup$ et $\liminf$}

\begin{definition}
	
Soit $(x_n)_{n\in\mathbb N} \in \mathbb R ^\mathbb N$ (plus généralement $\in\bar{\mathbb R}^\mathbb N$). Alors $s_n := \underset{k\geq n}{\sup} x_k$.

$s_n$ est décroissante (donc a une limite dans $\bar{\mathbb R})$.

Alors $\lim s_n =: \limsup x_n = \inf s_n$.

De même pour $\liminf x_n$. 
\end{definition}
\begin{propriete}
	$\lim x_n$ existe ssi $\liminf x_n = \limsup x_n$. Dans ce cas, $\lim x_n = \limsup x_n = \liminf x_n$.
\end{propriete}

\begin{demo}
	$\Leftarrow$ : $i_n \leq x_n \leq s_n$. On conclut par théorème d'encadrement.
	
	$\Rightarrow$ : Si $x_n \rightarrow l$ alors : $\forall \varepsilon > 0, \exists N \in \mathbb N, \forall n \geq N, l-\varepsilon \leq i_n \leq l \leq s_n \leq l+\varepsilon$.
	Donc $s_n \rightarrow l$ et $i_n \rightarrow l$.
\end{demo}

\begin{propriete}
	Si $y_n$ est une sous-suite de $x_n$, alors $\liminf x_n \le \liminf y_n \le \limsup y_n \le \limsup x_n$
\end{propriete}

Ainsi, si $l$ est valeur d'adhérence de $x_n$, alors $\liminf x_n \le l \le \limsup x_n$.
\begin{propriete}
	$\limsup x_n = -\liminf (-x_n)$
\end{propriete}

\begin{propriete}
	Il existe une sous-suite de $x_n$ qui converge vers $\limsup x_n$. Idem pour $\liminf x_n$.
\end{propriete}
\begin{demo}
	On choisit $k_n \ge n$ tel que $s_n - \frac{1}{n} \le x_{k_n} \le s_n$. $n \mapsto x_{k_n}$ converge vers $\limsup x_n$.
\end{demo}

\part*{Familles sommables}

On pose $(a_i)_{i\in I}$ famille de nombres positifs.

\begin{definition}
	$\sum_{i \in I} a_i := \underset{F \subset I \mathrm{fini}}{\sup}\sum_{i \in F}a_i$
\end{definition}
\begin{propriete}
	Si $\sum_{i \in I} a_i$ est fini, alors $\{i \in I, a_i\neq 0\}$ est au plus dénombrable.
\end{propriete}

\begin{demo}
	$\{i\in I, a_i \in \mathbb R \setminus \{0\}\} \subset \underset{k \in \mathbb N}{\bigcup}
	\underset{\# \le k\sum_{i\in I} a_i}{\underbrace{\{i \in I, a_i \ge \frac{1}{k}\}}}$
\end{demo}

À partir de maintenant, on considérera $I$ dénombrable.

\begin{propriete}
	Si $\sigma : \mathbb N \rightarrow I$ est une bijection, alors
	$\sum_{i\in I}a_i = \underset{n \rightarrow +\infty}{\lim}\sum_{k=1}^{n}a_{\sigma (k)} =: \sum_{k=1}^{+\infty}a_{\sigma(k)}$
\end{propriete}

\begin{demo}
	$\forall F \subset I$ fini, $\sigma ^{-1}(F)$ est fini donc majoré par un entier $N$.
	
	$\sum_{i\in F}a_i = \sum_{k\in \sigma^{-1}(F)}a_{\sigma(k)} \le \sum_{k=1}^N a_{\sigma(k)} \le \sum_{k=1}^{+\infty}a_{\sigma(k)}$
	
	Donc par passage au sup : $\sum_{i\in I} a_i \le \sum_{k=1}^{+\infty}a_{\sigma(k)}$.
	
	Réciproquement, $\sum_{k=1}^N a_{\sigma(k)} = \sum_{i\in\sigma(\llbracket 1,N \rrbracket)} a_i \le \sum_{i\in I}a_i$. On conclut par passage à la limite.
\end{demo}

\begin{corollaire}
	Si $(a_k) \in \mathbb R_+^{\mathbb N}, \sum_{k=1}^{+\infty}a_k=\sum_{k=1}^{+\infty}a_{\sigma(k)}$ et ce quel que soit $\sigma : \mathbb N \rightarrow \mathbb N$ bijection.
\end{corollaire}

En particulier dans le cas $I=\mathbb N^2, (a_{i,j})_{(i,j)\in I}\in \mathbb R_+^I$ :
\begin{propriete}
	$\sum_{(i,j)\in I}a_{i,j} = \sum_{i=1}^{+\infty}\left(\sum_{j=1}^{+\infty}a_{i,j}\right) =
	\sum_{j=1}^{+\infty}\left(\sum_{i=1}^{+\infty}a_{i,j}\right)$
\end{propriete}

\begin{demo}
	$F\subset I$ fini. Il existe $N \in \mathbb N$ tel que $F \subset \llbracket 1,N\rrbracket^2$. Donc $\sum_{(i,j)\in F}a_{i,j} \le \sum_{i=1}^{N}\sum_{j=1}^{N}a_{i,j} \le
	\sum_{i=1}^N\sum_{j=1}^{+\infty}a_{i,j} \le
	\sum_{i=1}^{+\infty}\sum_{j=1}^{+\infty}a_{i,j}$.
	
	Réciproquement, $\forall N\in\mathbb N, \forall M \in\mathbb N, \sum_{i=1}^N\sum_{j=1}^Ma_{i,j} \le
	\sum_{(i,j)\in \mathbb N^2}a_{i,j}$.
	
	Donc $(M\rightarrow +\infty)$, $\sum_{i=1}^N\sum_{j=1}^{+\infty}a_{i,j} \le
	\sum_{(i,j)\in \mathbb N^2}a_{i,j}$.
	
	Donc $(N\rightarrow +\infty)$, $\sum_{i=1}^{+\infty}\sum_{j=1}^{+\infty}a_{i,j} \le
	\sum_{(i,j)\in \mathbb N^2}a_{i,j}$.
\end{demo}

\part*{Séries absolument convergentes}

Soit $(a_i)_{i\in I}$ une famille de réels tels que $\sum_{i\in I}|a_i|$ soit finie.

On définit $a_i^+ := \max(a_i,0)$, $a_i^- := \max(-a_i, 0)$.

Donc $a_i^+ - a_i^- = a_i$ et $a_i^+ + a_i^- = |a_i|$.

\begin{propriete}
	$\sum_{i\in I}a_i^+ - \sum_{i\in I}a_i^- = \sum_{k=1}^{+\infty}a_{\sigma(k)}$ et ce quel que soit $\sigma : \mathbb N \rightarrow I$ bijection.
\end{propriete}

\begin{demo}
	$\sum_{i\in I}a_i^+ \le \sum_{i\in I} |a_i|$ donc la somme est finie. Idem pour $\sum_{i\in I}a_i^-$.
	
	$\sum_{k=1}^n a_{\sigma(k)} = \sum_{k=1}^n a_{\sigma(k)}^+ - \sum_{k=1}^n a_{\sigma(k)}^- \underset{n\rightarrow +\infty}{\rightarrow}
	\sum_{k=1}^{+\infty}a_{\sigma(k)}^+ - \sum_{k=1}^{+\infty}a_{\sigma(k)}^-$
\end{demo}
\begin{corollaire}
	Sous réserve de convergence absolue, on a :
	
	\[\sum_{k=1}^{+\infty}a_k = \sum_{k=1}^{+\infty} a_{\sigma(k)}\]
	
	\[\sum_{i=1}^{+\infty}\sum_{j=1}^{+\infty}a_{i,j} =
	\sum_{j=1}^{+\infty}\sum_{i=1}^{+\infty}a_{i,j}\]
\end{corollaire}

\part*{Vocabulaire}

\begin{definition}
	Soit $X$ un ensemble. On dit que $\mathcal A \subset \mathcal P(X)$ est :
	\begin{itemize}
		\item une algèbre (d'ensembles) si elle est stable par union finie, intersection finie et passage au complémentaire, contient $\emptyset$ et $X$.
		\item une tribu (ou $\sigma$-algèbre) si c'est une algèbre stable par réunion/intersection dénombrable.
	\end{itemize}
\end{definition}

\begin{exemple}
	\begin{itemize}
		\item $\mathcal P(X)$ est une tribu.
		\item $\{\emptyset, X\}$ est une tribu.
	\end{itemize}
\end{exemple}

Si on se donne une partition finie de $X$ : $X=X_1 \sqcup X_2 \dots \sqcup X_k$, alors l'ensemble des $A \subset X$ de la forme $A=\underset{n\in I \subset \llbracket 1,k \rrbracket}{\bigcup}X_n$ est une tribu finie.

\begin{lemme}
	Toute algèbre finie est associée à une partition finie.
\end{lemme}

\begin{demo}
	Soit $\mathcal A$ une algèbre finie.
	
	$\forall x \in X, A(x) := \underset{x\in A}{\underset{A\in \mathcal A}{\bigcap}} A$.
	
	Pour $x$ et $y$ donnés, soit $A(x) = A(y)$, soit $A(x) \cap A(y) = \emptyset$.
	
	Fixons $x\in X, B\in\mathcal A$.
	\begin{itemize}
		\item Soit $x\in B$ et alors $A(x)\subset B$.
		\item Soit $x\in ^c\!\!B$ et alors $A(x)\subset ^c\!\!B$ i.e. $A(x)\cap B = \emptyset$
	\end{itemize}
	On conclut avec $B=A(y)$.
\end{demo}

\begin{definition}
	Si $\mathcal A$ est une algèbre de $X$ et $m:\mathcal A \rightarrow [0,+\infty]$ une fonction.
	
	On dit que $m$ est une \emph{mesure additive} si :
	\begin{itemize}
		\item $m(\emptyset) = 0$
		\item $m(A\sqcup B) = m(A) + m(B) \qquad (A\cap B = \emptyset)$
	\end{itemize}
\end{definition}

\begin{definition}
	Si $\mathcal T \subset \mathcal P(X)$ est une tribu, $m:\mathcal T \rightarrow [0,+\infty]$ est une \emph{mesure} si :
	\begin{itemize}
		\item $m(\emptyset) = 0$
		\item $m(\underset{i\in I}{\bigsqcup}A_i) = \sum_{i\in I}m(A_i)$ pour $(A_i)_{i\in I}$ famille dénombrable disjointe.
	\end{itemize}
\end{definition}
\begin{rem}
	Toute mesure est une mesure additive.
\end{rem}

\begin{rem}
On appelle parfois les mesures "mesures $\sigma$-additives".
\end{rem}

\begin{rem}
	Lorsque $m:\mathcal A \rightarrow [0,+\infty]$ est une mesure additive sur une algèbre, les propriétés suivantes sont équivalentes :
	\begin{enumerate}
		\item Si $A_i\in\mathcal A$ sont disjoints, $(A_i)$ dénombrable, $\underset{i\in I}{\bigsqcup}A_i \in \mathcal A$, alors $m(\underset{i\in I}{\bigsqcup}A_i) = \sum_{i\in I}m(A_i)$
		\item Si $A,A_i\in\mathcal A$, $A\subset \underset{i\in I}{\bigcup}A_i$, alors $m(A) \le \sum_{i\in I}m(A_i)$.
	\end{enumerate}
Dans ce cas, on dit que $m$ est $\sigma$-additive.
\end{rem}

\begin{demo}
	$(1) \Rightarrow (2)$ :
	
	Soit $A_i \in \mathcal A$. On définit $\tilde{A_i}$ par : $\tilde A_1 = A_1, \dots \tilde A_n = A_n\setminus \tilde A_{n-1} \quad \forall n \ge 1$
	
	Alors $\bigcup A_i = \bigsqcup \tilde A_i$.
	
	Si $A \subset \bigcup A_i$, alors $A \subset \bigsqcup \tilde A_i$. Alors $A=\bigsqcup(A\cap\tilde A_i)$.
	
	Donc $m(A)=m(\bigsqcup (\tilde A_i \cap A)) \le \sum m(A_i)$.

	$(2) \Rightarrow (1)$ :
	
	Si $A=\bigsqcup A_i \overset{(2)}{\Rightarrow} m(A) \le \sum m(A_i)$.
	
	$A\supset\overunderset{n}{i=1}{\bigsqcup}A_i$ quel que soit $n$.
	
	Donc $m(A) \ge \sum_{i=1}^n m(A_i)$. Donc ($n \rightarrow +\infty$), $m(A) \ge \sum_{i=1}^{+\infty}m(A_i)$.
	
\end{demo}

\begin{definition}
	Soit $f:\Omega \rightarrow X$ une application. Si $\mathcal A$ est une algèbre (ou une tribu) sur $\Omega$, alors on définit l'algèbre (tribu) image par :
	
\[ f_*\mathcal A = \{A\subset X, f^{-1}(A)\in \mathcal A\}\]
	Si $\mathcal A$ est une algèbre (tribu) sur $X$, alors \[f^*\mathcal A = \{f^{-1}(A), A\in\mathcal A\}\] est une algèbre (tribu) sur $\Omega$.
\end{definition}

La vérification du fait que $f^*\mathcal A$ et $f_*\mathcal A$ est une algèbre (tribu) découle des propriétés des préimages :

\begin{align*}
	&f^{-1}(A\cap B) = f^{-1}(A)\cap f^{-1}(B) \\
	&f^{-1}(A \cup B) = f^{-1}(A) \cup f^{-1}(B) \\
	&f^{-1}(A \setminus B) = f^{-1}(A) \setminus f^{-1}(B) 
\end{align*}

\begin{definition}
	Si $f:(\Omega, \mathcal A, m) \rightarrow X$ est une application, on définit la mesure image (ou la loi) comme la mesure :
	\[ (f_*m)(Y) := m(f^{-1}(Y))\] définie sur $f_*\mathcal A$.
\end{definition}

\begin{definition}
	$m$ est dite finie ssi $m(X) < +\infty$
\end{definition}
\begin{definition}
	$m$ est dite de probabilité ssi $m(X)=1$
\end{definition}

\begin{definition}
	$f:(\Omega,\tau) \rightarrow (X,\mathcal T)$ est dite mesurable si : \[\forall Y \in \mathcal T, f^{-1}(Y) \in \tau\] i.e.
	\begin{align*}
		&f_*\tau \supset \mathcal T \\
		&f^*\mathcal T \subset \tau
	\end{align*}
\end{definition}

\begin{exo}
	Soit $\Omega, X$ des ensembles, $\mathcal T$ une tribu sur $X$. Soit $f:\Omega \rightarrow X$ une application, $g:\Omega\rightarrow X$ une application à valeurs dans un ensemble fini $Y$. Alors $g$ est $f^*\mathcal T$ mesurable ssi $\exists h:(X,\mathcal T) \rightarrow (Y,\mathcal P(Y))$ mesurable telle que $g=h \circ f$. i.e. "$g$ est $f$-mesurable ssi $g$ ne dépend que de $f$".
\end{exo}

\part*{Modélisation d'une expérience aléatoire finie (ex : jets de dés)}

Soit $Y$ un ensemble fini représentant les issues possibles. Il y a 2 manières de représenter un tirage aléatoire sur $Y$.
\begin{enumerate}
	\item On se donne une mesure de probabilité sur $(Y,\mathcal P(Y))$. Pour ceci, il suffit de donner $p:Y\rightarrow[0,1]$ tel que $\sum_{y\in Y}p(y) = 1$. On note $P$ la mesure de probabilité ainsi créée.
	\item On se donne un espace de probabilité abstrait $(\Omega,\mathcal T, \mathbb P)$ et une application mesurable $f:\Omega \rightarrow Y$ telle que $f_*\mathbb P=P$.
\end{enumerate}
Pour passer de 1. à 2., il suffit de prendre $\Omega = Y$, $\mathcal T = \mathcal P(Y)$, $\mathbb P = P$, $f=\mathrm {id}$.

L'expérience aléatoire consistant à jeter un nombre fini $k$ de dés de valeurs possibles $Y_1, \dots, Y_k$ est simplement une expérience aléatoire à valeurs dans le produit $Y=Y_1 \times Y_2 \times \dots\times Y_k$.

La description en termes de variables aléatoires consiste donc à se donner une application mesurable $f : (\Omega, \mathcal T, P) \rightarrow Y$, c'est à dire $k$ applications mesurables $f_i : (\Omega, \mathcal T, P) \rightarrow Y_i$, définies sur \emph{le même espace de probabilités}.

\begin{definition}
	La loi de $f$ (qui est une probabilité sur $Y$) est dite \emph{loi jointe}. Les lois des $f_i$ (qui sont des probabilités sur $Y_i$) sont dites \emph{lois marginales}.
\end{definition}

\begin{rem}
	La loi jointe détermine les lois marginales, qui peuvent se décrire explicitement par $m_i(y_i)=\underset{y_1,\dots,y_{i-1},y_{i+1},\dots,y_k}{\sum} m(y_1,\dots,y_k$.
	
	Plus abstraitement, ce soint les mesures images $m_i=(\Pi_i)_*m_i$ où $\Pi_i : Y \rightarrow Y_i$ est la projection.
\end{rem}

\begin{rem}
	La loi jointe est déterminée par $|Y_1|\times\dots\times |Y_k| - 1$ nombres réels ($-1$ à cause de la contrainte $\sum p = 1$).
	
	Les lois maginales sont déterminées par $|Y_1| + \dots + |Y_k| - k$ nombres réels, ce qui est beaucoup moins.
\end{rem}

Si on se donnes les marginales $m_1, \dots, m_k$, ilm existe de nombreuses lois jointes qui engendrent ces marginales. L'une d'entre elles est particulièrement intéressante : la loi produit $m((y1_,\dots,y_k))=m_1(y_1) \cdot \dots \cdot m_k(y_k)$, qui correspond (par définition) à des expériences indépendantes.

\begin{definition}
	\begin{itemize}
		\item Les événement $A,B$ dans un espace de probabilité $(\Omega,\mathcal T, P)$ sont dits indépendants si $P(A\cap B) = P(A)P(B)$.
		\item Si $(X_i, \mathcal T_i)_{1\le i\le k}$ sont des espaces mesurables (c'est à dire munis de tribus $\mathcal T_i$), les variables aléatoires (applications mesurables) $f_i : (\Omega,\mathcal T, P) \rightarrow (X_i,\mathcal T_i)$ sont dites \emph{indépendantes} si $\forall Z_i \in \mathcal T_i, P(f_1\in Z_1, \dots, P_k \in Z_k) = P(f_1 \in Z_i) \cdot \dots \cdot P(f_k \in Z_k)$
	\end{itemize}
\end{definition}

\begin{propriete}
	Les événements $A$ et $B$ sont indépendants ssi les variables aléatoires $\mathds{1}_A, \mathds{1}_B : (\Omega, \mathcal T, P) \rightarrow \{0,1\}$ le sont.
\end{propriete}

\begin{demo}
	Il suffit de montrer que $\!\!^cA$ et $B$ sont indépendants (le reste est évident ou vient par symétrie).
	\begin{align*}
		P(\!\!^cA\cap B) &= P((\Omega \setminus A) \cap B) \\
		&= P(B \setminus A\cap B) \\
		&= P(B) - P(A\cap B) \\
		&= P(B) - P(A)P(B) \\
		&= (1-P(A))P(B) \\
		&= P(\!\!^cA)P(B)
	\end{align*} 
\end{demo}

\begin{definition}
	Les événements $A_1,\dots,A_k$ sont dits indépendants si $\mathds 1_{A_1} \dots \mathds 1_{A_k} : \Omega \rightarrow \{0,1\}$ le sont.
\end{definition}

\begin{rem}
	Il ne suffit pas d'avoir l'indépendance deux à deux ou $P(A_1 \cap \dots \cap A_k) = P(A_1) \cdot \dots \cdot P(A_k)$.
\end{rem}

\begin{propriete}
	Il suffit d'avoir $P(A_{i_1} \cap \dots \cap A_{i_k}) = P(A_{i_1}) \cdot \dots \cdot P(A_{i_k})$, et ce $\forall \{i_1, \dots, i_k\} \subset \llbracket 1,k \rrbracket$.
\end{propriete}

\begin{demo}
	Il faut montrer que
	\[ (*) \quad P(B_1 \cap \dots \cap B_k) = P(B_1) \cdot \dots \cdot P(B_k) \forall B_i \in \{\emptyset, A_i, \!\!^cA_i, \Omega\}	\]
	Il découle de l'hypothèse que c'est vrai pour $B_i \in \{ \emptyset, A_i, \Omega\}$.
	
	Il suffit donc de constater que $(*)$ implique $P\!\!^cB_1 \cap B_2 \cap \dots \cap B_k) = P(\!\!\c B_1)P(B_2) \cdot \dots \cdot P(B_k)$, ce qui se montre comme ce-dessus. On conclut par récurrence finie.
\end{demo}

\begin{exemple}
	Tirage non indépendant :
	
	On tire $-$ chiffres dans $\llbracket 1,6 \rrbracket$, en leur imposant d'être distincts. La loi jointe est donc : $P(y_1, \dots, y_6) = \begin{cases}
		0 &\mathrm{\ si\  non\  distincts} \\
		\frac{1}{6!} &\mathrm{\ si\ distincts}
	\end{cases}$.

	Les lois marginales sont : $P_1(y_1):= \underset{y_2,\dots,y_k}{\sum} P(y_1,\dots,y_k) = \frac{5!}{6!} = \frac{1}{6}$. Les lois marginales sont donc les mêmes que pour un tirage indépendant !
\end{exemple}
\begin{definition}
	On dit que $f_i,\ i\in I$ sont indépendantes si $f_i,\ i\in F$ le sont pour tout $F\subset I$ fini.
\end{definition}

\part*{Modélisation d'une suite infinie de jets dés indépendants}

Donnons-nous une suite infinie d'espaces de probabilités finis $(Y_i,P_i)$ (la tribu est $\mathcal P(Y_i)$).

Pour chaque $n$, on a vu que l'on peut trouver des variables aléatoires indépendantes $f_i:(\Omega_n,\mathcal T_n, P_n) \rightarrow Y_i$ de loi $P_i$.

Question : peut-on prendre $(\Omega_n,\mathcal T_n, P_n)$ indépendant de $n$?

\begin{theorem}
	Il existe un espace de probabilité $(\Omega,\mathcal T, P)$ et une suite de variables aléatoires $f_i : \Omega \rightarrow Y_i$ qui sont indépendantes et de loi $P_i$.
\end{theorem}

\begin{rem}
	Les variables aléatoires $f_i$, $i\in \mathbb N$ sont indépendantes ssi $f_1, \dots, f_n$ le sont pour tout $n$.
\end{rem}

L'hypothèse d'indépendance consiste donc à dire que, pour tout $n$ et pour tout $(y_1, \dots, y_n) \in Y_1 \times \dots \times Y_n$, l'événement $\{f_1=y_1,\dots,f_n=y_n\}$ est mesurable ()$\in\mathcal T$) et de mesure $P(f_1=y_1, \dots f_n=y_n) = P_1(y_1)\cdot\dots\cdot P_n(y_n)$.

En termes de loi, ceci implique que $\{y_1\}\times \dots \{y_n\} \times Y_{n+1} \times \dots$ est mesurable sur $X:=\prod Y_i$ et que sa mesure est $m(\{y_1\}\times \dots \times Y_{n+1}) = P_1(y_1) \cdot \dots \cdot P_n(y_n)$.

\section*{Introduction de l'algèbre $\mathcal A_\infty$ engendrée par les cylindres finis}

Sur le produit $X = \prod Y_i$, pour $n$ fixé, les ensembles de la forme $\{y_1\}\times \dots \times \{y_n\}\times Y_{n+1} \times \dots$ forment une partition finie (ce sont les cylindres finis), qui engendre une algèbre finie $\mathcal A_n$ (qui est donc aussi une tribu).

C'est l'algèbre engendrée par les $n$ premières coordonnées. En effet si $\Pi : X \rightarrow Y_1 \times \dots \times Y_n$ est la projection, alors $\mathcal A_n = \Pi^* (\mathcal P(Y_1 \times \dots \times Y_n))$.

Cette algèbre décrit les parties de $X$ qui peuvent être décrites en termes des $n$ premières coordonnées.

On a $\mathcal A_n \subset \mathcal A_{n+1}$. On note $\mathcal A_\infty = \underset{n\ge 1}{\bigcup} \mathcal A_n$.

$\mathcal A_\infty$ est donc l'algèbre des parties de $X$ qui dépendent d'un nombre fini de coordonnées. C'est l'algèbre engendrée par les cylindres finis.

Contrairement aux $\mathcal A_n$, $\mathcal A_\infty$ est infinie et ce n'est pas une tribu !

L'hypothèse d'indépendance des $f_i$ implique que la loi $m$ doit être définie sur $\mathcal A_\infty$, et qu'elle y est déterminée par la relation \[ (*) \quad m(\{y_1\}\times \dots \times \{y_n\} \times Y_{n+1} \times \dots ) = P_1(y_1) \cdot\dots\cdot P(y_n)\]

\begin{theorem}
	Il existe sur $X = \prod Y_i$ une tribu $\tau$, qui contient $\mathcal A_\infty$, et une mesure $m$ sur $\mathcal T$ qui vérifie $(*)$.
\end{theorem}

On vient en fait de voir que le théorème 1. implique le théorème 2.
Réciproquement, il suffit de prendre $\Omega = X, \mathcal T = \tau, P=m, f=\text{projection}$.

Pour démontrer l'utilité du théorème 2., donnons des exemples d'ensembles qu'il est naturel de considérer et qui sont dans $\tau$ mais pas dans $\mathcal A_\infty$. On suppose $Y_i \subset \mathbb R$

\begin{exemple}
	L'ensemble $\{(y_i)\in X, \frac{y_1 + \dots y_n}{n} \rightarrow l\}$ est mesurable. En effet, il s'écrit : $\underset{k\ge 1}{\bigcap}\ \underset{n\in \mathbb N}{\bigcup}\ \underset{m \ge n}{\bigcap} \{\left| \frac{y_1+ \dots y_n}{n} - l \right| \le \frac{1}{k}\}, \text{ i.e. } \forall k \ge 1, \exists n\in \mathbb N, \forall m\ge n, \dots$. 
	Chacun des ensembles est dans $\mathcal A_\infty$ donc l'ensemble considéré est dans $\tau$.
\end{exemple}

\section*{$\liminf$ et $\limsup$ d'ensembles}
	Si $A_n$ est une suite d'ensembles, on note :
	\[ \liminf A_n = \underset{n}{\bigcup}\ \underset{m\ge n}{\bigcap} A_m = \{A_i \text{ APCR}\}\]
	\[ \limsup A_n = \underset{n}{\bigcap}\ \underset{m\ge n}{\bigcup} A_m = \{A_i \text{ infinitely often (i.o.)}\}\]
	
	Si $\tau$ est une tribu, que les $A_n \in \tau$, alors $\limsup A_n \in \tau$ et $\liminf A_n \in \tau$.
	
	\begin{propriete}
		\begin{itemize}
			\item $\liminf A_n \subset \limsup A_n$
			\item $\liminf \!\!^cA_n= \!\!^c(\limsup A_n)$
		\end{itemize}
	\end{propriete}
	\begin{demo}
		$\forall m,M, \quad \underset{n\ge m}{\bigcap} A_n \subset \underset{n \ge M} A_n$.
		Donc $\underset{n\ge m}{\bigcap} A_n \subset \limsup A_n$, donc $\liminf A_n \subset limsup A_n$
	\end{demo}
\begin{exo}
	$\limsup \mathds 1_{A_n} = \mathds 1_{\limsup A_n}$
\end{exo}

\begin{exemple}
	On considère un tirafe aléatoire indépendant $f_n \in {-1, 1}^{\mathbb N}$, ce que l'on voit comme un jeu de hasard (le joueur gagne ou perd $1$ à chaque étape). Étant donnée la richesse initiale $r_0$ et un objectif $R$, on considère l'événement \{le joueur atteint la richesse $R$ avant de se ruiner\}.
	
	Il s'écrit $\underset{n \ge 1}{\bigcup} \{y_1 + \dots y_n \ge -r_0 \quad \forall k < n \text{ et } y_1 + \dots + y_k = R-r_0\}$.
	
	C'est une réunion dénombrable d'éléments de $\mathcal A_\infty$
\end{exemple}

Le théorème 2 sera déductible du théorème suivant :
\begin{theorem}[Hahn-Kolmogorov]
	Soit $\mathcal A$ une algèbre d'ensembles sur $X$. Soit $\underline m$ une mesure de probabilité additive sur $\mathcal A$, qui vérifie la propriété de $\sigma$-additivité.
	
	Alors il existe une tribu $\tau$ contenant $\mathcal A$, et une mesure de proba $m$ sur $\tau$ qui prolonge $\underline m$. De plus, on peut prendre : $m(B) = \underset{B \subset \bigcup A_i} \inf \  \underset{i \in \mathbb N}{\sum} \underline m (A_i)$, où le $\inf$ est pris sur les recouvrements dénombrables de $B$ par des éléments de $\mathcal A$.
\end{theorem}

Pour démontrer le théorème 2, on va appliquer le théorème 3 avec $\mathcal A = \mathcal A_\infty$, et $\underline m$ la mesure additive déterminée par $\underline m (\{y_1\}\times \dots \times \{y_n\} \times Y_{n+1} \times \dots) = P_1(y_1) \dots P_n(y_n)$.

Il nous suffit donc de vérifier que cette mesure additive a la propriété de $\sigma$-additivité.

\begin{propriete}
	Toute mesure additive sur $\mathcal A_\infty$ est $\sigma$-additive.
\end{propriete}

\begin{demo}
	Soient $A \in \mathcal A_\infty$ et $A_i \in \mathcal A_infty$ tel que $A \subset \underset i \bigcup A_i$, alors $\exists n, A \subset \bigcup_{i=1}^n A_i$.
	
	\begin{itemize}
		\item méthode savante : c'est la compacité de $A$ dans $X$ muni de la topologie produit (les $A_i$ sont ouverts et compacts)
		\item à la main :
		On pose $B_n = A \setminus \bigcup_{i=1}^n A_n$. On veut montrer que $\exists n, B_n = \emptyset$, sachant que $\underset {n\ge 0}\bigcap B_n = \emptyset$.
		
		On suppose que $B_n \ne \emptyset, \forall n$. On note $B_n(y_1) := \Pi_1^{-1}(y_1) \cup B_n$, ce sont les éléments de $B_n$ qi commencent par $y_1$.
		
		Pour chaque $y_1$, $n \mapsto B_n(y_1)$ est décroissante. Comme $B_n = \underset {y_1 \in Y_1} \bigcup B_n(y_1)$ (union finie) (et $B_n \ne \emptyset$), il existe $y_1$ tel que les $B_n(y_1)$ sont tous non vides.
		
		On fixe maintenant un tel $y_1$ et on reprend le même raisonnement sur $y_2$, puis... On obtient de la sorte une suite $y$.
		
		Ainsi, il existe une suite $(y_1, \dots) \in B_n \forall n$ car $\forall n, \exists k_n, B_n\in\mathcal A_{k_n}$.
		
		Ainsi, $\forall n, B_n \ni y$ donc $\bigcap B_n \ne \emptyset$. Absurde.
	\end{itemize}
\end{demo}

\begin{propriete}
	Dans le contexte du théorème d'Hahn-Kolmogorov, $m^* : \mathcal P(X) \rightarrow [0,\infty]$ est une \emph{mesure extérieure}, c'est à dire que $m^*(\emptyset) = 0$, $m^*$ est croissante, et $m^*\left(\underset{i\in \mathbb N} \bigcup Z_i\right) \le \underset{i\in\mathbb N}\sum m^*(Z_i), \forall Z_i$.
\end{propriete}

\begin{demo}
	Démontrons la dernière propriété. Fixons $\varepsilon > 0$. Pour tout $i$, il existe un recouvrement $A_{i,j}, j\in \mathbb N$ de $Z_i$ tel que $\underset j \sum \underline m(A_{i,j}) \ge m^*(Z_i) \ge \underset j \sum \underline m (A_{i,j}) - \varepsilon 2^{-i}$, alors $A_{i,j}, i\in \mathbb N, j \in \mathbb N$ est un recouvrement de $\bigcup Z_i$, et $m^*(\bigcup Z_i) \le \underset {i,j} \sum \underline m (A_{i,j}) \le \underset {i \ge 1} \sum (m^*(Z_i) + \varepsilon 2^{-1}) \le \varepsilon + \underset {i \ge 1} \sum m^*(Z_i)$.
\end{demo}

\begin{demo}
	\emph{Démonstration du théorème d'Hahn-Kolmogorov}
	Deux étapes : \begin{enumerate}
		\item $m^*|_{\mathcal A} = \underline m$
		Si $A \subset \underset i \bigcup A_i$, alors $\underline m (A) \le \sum \underline m (A_i)$ par $\sigma$-additivité de $\underline m$. En prenant l'inf, on obtient $\underline m(A) \le m^*(A)$. L'inégalité réciproque s'obtient en considérant le recouvrement trivial $A_1 = A, A_2=A_3 = \dots = \emptyset$.
		\item On dit que $Y \subset X$ est mesurable si, pour tout $\varepsilon > 0, \exists A \in \mathcal A$ tel que $m^*(Y \Delta A) \le \varepsilon$. Alors l'ensembre $\mathcal T$ des parties mesurables est une algèbre.
		\begin{demo}
			\begin{itemize}
				\item si $m^*(Y \Delta A) \le \varepsilon$, alors $m^*(^c\!\!Y \cap ^c\!\!A) \le \varepsilon$, donc $\mathcal T$ est stable par complément.
				\item Soient $Y, Z$ mesurables et $A, B$ tels que $m^*(Y \Delta A) \le \varepsilon, m^*(Z \Delta B) \le \varepsilon$ alors $m^*((Y \cup Z) \Delta (A \cup B)) \le 2\varepsilon$ car $(Y \cup Z)  \Delta (A \cup B) \subset (Y \Delta A) \cup (Z \Delta B)$.
			\end{itemize}
		\end{demo}
		\item $m^*$ est une mesure additive sur $\mathcal T$.
		\begin{demo}
			$Y, Z$ disjoints, $A,B$ comme ci-dessus. 
			
			$(A\cap B) = (Y \cup (A \setminus Y)) \cap (Z \cup (B \setminus Z)) \subset Y \cap Z \cup (B \setminus Z) \cup (A \setminus Y)$ 
			
			donc $\underline m (A \cap B) \le 2\varepsilon$
			
			$A \cup B = (Y \cup (A \setminus Y)) \cup (Z \cup (B \setminus Z)) \subset Y \cup Z \cup (A \setminus Y) \cup (B \setminus Z)$
			
			$\underline m (A \cup B) \le m^*(Y \cup Z) +2\varepsilon$
			
			et $\underline m (A \cup B) = \underline m(A) + \underline m(B) - \underline m(A \cap B) \ge \underline m(A) + \underline m(B) -2\varepsilon \ge m^*(Y) - \varepsilon + m^*(Z) - \varepsilon - 2\varepsilon$.
			
			Finalement, $m^*(Y) + m^*(Z) \le m^*(Y \cup Z) + 6\varepsilon$ 
		\end{demo}
	Comme $m^*$ est une mesure extérieure et une mesure additive sur l'algèbre $\mathcal T$, elle a la propriété de $\sigma$-additivité.
	\item $\mathcal T$ est une tribu.
	\begin{demo}
		$Y_i \in \mathcal T$. On veut montrer que $Y_\infty := \underset i \bigcup Y_i \in \mathcal T$. On peut supposer que les $Y_i$ sont disjoints. Alors $\forall n, m^*(\underset{i = 1}{\overset{n} \bigsqcup} Y_i) = \underset{i = 1}{\overset{n} \sum} m^*(Y_i) \le m^*(X) = 1$. Donc la série $\sum m^*(Y_i)$ converge, donc $\forall \varepsilon, \exists n, \underset{i = n+1}{\overset{+\infty} \sum} m^*(Y_i) \le \varepsilon$.
					
		Alors en posant $Z= \underset{i = 1}{\overset{n} \bigcup} Y_i$, on a $m^*(Y_\infty \setminus Z) \le \varepsilon$, $Z \subset Y_\infty$. Ensuite, on prend $A \in \mathcal A$ tel que $m^*(A\setminus Z) \le \varepsilon, m^*(Z\setminus A) \le \varepsilon$. On obtient $A \setminus Y_\infty \subset A \setminus Z, Y_\infty \setminus A \subset (Z \setminus A) \cup (Y_\infty \setminus Z)$.
	\end{demo}
	\end{enumerate}
\end{demo}

\textbf{Complément :} on aurait pu donner une autre preuve du théorème 3 basée sur un résultat général sur les mesures extérieures. Lorsque $m^*$ est une mesure extérieure, on dit que $Y \subset X$ est $m^*$-mesurable si

$\forall Z \subset X, m^*(Z) = m^*(Z \cap Y) + m^*(Z \cap ^c\!\!Y)$.

\begin{theorem}[Carathéodory]
	Si $m^*$ est une mesure extérieure, l'ensemble $\mathcal T$ des parties $m^*$-mesurables est une tribu, et $m^*|_{\mathcal T}$ est une mesure.
\end{theorem}
\begin{rem}
	Dans le cas du théorème de Hahn, la tribu $\mathcal T$ est la même que celle introduite dans la démonstration précédente.
\end{rem}

\begin{demo}
	\emph{Carathéodory $\Rightarrow$ Hahn-Kolmogorov}
	
	Il suffit de montrer que les éléments de $\mathcal A$ sont $m^*$-mesurables, et que $m^*|_{\mathcal A} = \underline m$.
	\begin{itemize}
		\item $m^*(A) \le \underline m(A) \forall A \in \mathcal A$
		\item $m^*(A) \ge \underline m (A) \forall A \in \mathcal A$. En effet, si $A \subset \underset i \bigcup A_i$, on peut supposer les $A_i$ disjoints. Alors par $\sigma$-additivité de $\underline m$ sur $\mathcal A$ : $\underline m(A) = \underset i \sum \underline m(A_i) \ge m^*(A)$.
		\item Soit $A \in \mathcal A$ et $Zin \mathcal P(X)$. On considère un recouvrement $A_i$ de $Z$.
		
		$\underset i \sum \underline m(A_i) = \underset i \sum \underline m (A_i \cap A) + \underline m(A_i \cap ^c\!\!A) \ge m^*(Z \cap A) + m^*(Z \cap ^c\!\!A)$.
		
		On prend l'inf : $m^*(Z) \ge m^*(Z \cap A) + m^*(Z \cap ^c\!\!A)$. L'autre inégalité découle de la sous-additivité.
	\end{itemize}
\end{demo}

\begin{demo}
	\emph{Carathéodory}
	
	\begin{enumerate}
		\item $\mathcal T$ est une algèbre. 
		\begin{demo} On a $\emptyset \in \mathcal T, X \in \mathcal T$, et stabilité par complément de manière triviale.
		
		$A,B \in \mathcal T \Rightarrow \forall Y, m^*(Y) = m^*(Y\cap A) + m^*(Y \cap ^c\!\!A) = m^*(Y \cap A \cap B) + m^*(Y \cap A \cap ^c\!\!B) + m^*(Y \cap ^c\!\!A \cap ^c\!\!B) + m^*(Y \cap ^c\!\!A \cap B)$.
		
		\begin{rem}
			$^c\!\!(A \cap B) = (^c\!\!B \cap A) \cup (B \cap ^c\!\!A) \cup (^c\!\!A \cap ^c\!\!B)$
		\end{rem}
		Donc $m^*(Y) \ge m^*(Y \cap (B \cup A)) + m^*(Y \cap ^c\!\!(B \cap A))$
		\end{demo}
		\item $m^*$ est additive sur $\mathcal T$
		\begin{demo}
			$A,B \in \mathcal T$, $A\cap B = \emptyset$.
			
			$m^*(A\cup B) = m^*((A\cup B)\cap A) + m^*((A \cup B) \cap ^c\!\!A) = m^*(A) + m^*(B)$
		\end{demo}
		\item $\mathcal T$ est une tribu.
		\begin{demo}
			soit $A_n$ une suite d'éléments deux à deux disjoints de $\mathcal T$. Posons $B_n =\bigcup_{k=1}^n A_n$ et $B_\infty = \bigcup_{k=1}^\infty A_n$.
			
			$\forall Y \subset X, m^*(Y \cap B_n) = m^*(Y \cap B_n \cap A_n) + m^*(Y \cap B_n \cap ^c\!\!A_n) = m^*(Y \cap A_n) + m^*(Y \cap B_{n-1})$.
			
			Donc $m^*(Y\cap B_n) = \sum_{k=1}^n m^*(Y \cap A_k)$.
			
			Alors $m^*(Y) = m^*(Y) = m^*(Y \cap B_n) + m^*(Y \cap ^c\!\!B_n) \ge \sum_{k=1}^n m^*(Y \cap A_k) + m^*(Y \cap ^c\!\!B_{infty})$.
			
			À la limite : $m^*(Y) \ge \sum_{n=1}^\infty m^*(Y \cap A_n) + m^*(Y \cap ^c\!\!B_{\infty})$
		\end{demo}
	\end{enumerate}
\end{demo}

\end{document}